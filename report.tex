\documentclass[a4paper]{article}
\usepackage{fontspec}
\usepackage{xunicode}
\usepackage{xltxtra}
\usepackage{xcolor}
\usepackage{textcomp}
\setromanfont{Ubuntu}
\setsansfont{Ubuntu}
\setmonofont{Ubuntu}
\setmainfont{Ubuntu}
\usepackage{fancyhdr} % Required for custom headers
\usepackage{lastpage} % Required to determine the last page for the footer
\usepackage{extramarks} % Required for headers and footers
\usepackage{color} % Required for custom colors
\usepackage{graphicx} % Required to insert images
\usepackage{listings} % Required for insertion of code
\usepackage{courier} % Required for the courier font
\usepackage{lipsum} % Used for inserting dummy 'Lorem ipsum' text into the template
\usepackage{authblk}
\usepackage{hyperref}
\usepackage{mdframed}
\usepackage{subfigure}
% Margins
\topmargin=-0.45in
\evensidemargin=0in
\oddsidemargin=0in
\textwidth=6.5in
\textheight=9.0in
\headsep=0.25in

\linespread{1.1} % Line spacing


\renewcommand\headrulewidth{0.4pt} % Size of the header rule
\renewcommand\footrulewidth{0.4pt} % Size of the footer rule

\setlength\parindent{0pt} % Removes all indentation from paragraphs



\newcommand{\horrule}[1]{\rule{\linewidth}{#1}} % Create horizontal rule command with 1 argument of height


\title{	
\normalfont \normalsize 
\textsc{Project Εξαμήνου} \\ [30pt] % Your university, school and/or department name(s)
\horrule{0.5pt} \\[0.4cm] % Thin top horizontal rule
\huge Δομές Δεδομένων \\ % The assignment title
\horrule{2pt} \\[0.5cm] % Thick bottom horizontal rule
\vspace*{4\baselineskip}
}

%\vspace*{4\baselineskip} % Whitespace between location/year and editors

\author[]{Δουράτσος Ιωάννης  \thanks{\href{mailto:douratsos@ceid.upatras.gr}{douratsos@ceid.upatras.gr}} 4978 }
\affil[]{Τμήμα Μηχανικών Η/Υ και Πληροφορικής, Πανεπιστήμιο Πατρών}



\date{\normalsize\today} % Today's date or a custom date

\begin{document}
\newpage
\maketitle % Print the title
\newpage
\tableofcontents
\vspace*{1\baselineskip}
\newpage
\section{Γενικές Πληροφορίες}
Η υλοποίηση του project έχει γίνει σε python 3.3. Απαραίτητη είναι και η ύπαρξη του matplotlib module καθώς αυτό αναλαμβάνει το plotting των διαγραμμάτων χρόνου.
Αν εξαιρεθεί το plotting, οποιοσδήποτε python interpreter αρκεί.
Για την ευκολότερη εκτέλεση των test, υπάρχουν δύο εναρκτήρια αρχεία: το setup.sh και το create.py. Το setup.sh αναλαμβάνει την εγκατάσταση των module που είναι αναγκαία και το create.py φτιάχνει μια μικρή βάση από βιβλία τα οποία και θα χρησιμοποιηθούν ως δείγμα για την αναζήτηση πάνω στη πραγματική βάση, δηλαδή το books.csv.
Τα Book, Author και Books έχουν όλα υλοποιηθεί ως κλάσεις, ενώ συγκεκριμένα το Books περιέχει μια list από book και όλες τις κατάλληλες συναρτήσεις για αυτά.


\section{Μέρος Α}
Για αυτό το ερώτημα εκτυπώνεται ένα menu από το οποίο ο χρήστης μπορεί να διαλέξει την συνάρτηση που θέλει να εκτελέσει πάνω στα Books. Όπως είπαμε και πριν, όλα τα book βρίσκονται στη μνήμη σε μια list άπαξ και γίνουν load μέσω της επιλογής 1. 
Η επιλογή αυτή απαιτεί ένα αρχείο σε μορφή csv, ενώ από default διαβάζει το αρχείο books.csv που βρίσκεται στον ίδιο φάκελο και είναι αυτό που είχε δωθεί στο forum του μαθήματος. Όπου χρειάζεται αναζήτηση, το κάνουμε με μια for πάνω στη λίστα ώστε να έχουμε γραμμικό χρόνο αναζήτησης.

\section{Μέρος Β}
Χρησιμοποιείται η κλασική μέθοδος δυαδικής αναζήτησης, η οποία είναι υλοποιημένη στη συνάρτηση binary\_search μέσα στη κλάση Books. Για τη χρήση της δυαδικής αναζήτησης υπάρχει η ανάγκη να διατηρείται η λίστα των βιβλίων ταξινομημένη στο σύστημα μας, για αυτό δημιουργούμε ένα ταξινομένο αντίγραφο της, την arrsorted. Σε κάθε insert/delete γίνεται ξανά sorting της λίστας.

\section{Μέρος Γ}
Τα digital tries βρίσκονται υλοποιημένα στο module trie.py. Θα χρειαστούμε δύο διαφορετικά trie για το μέρος αυτό, ένα για να οργανώσουμε με βάση τον τίτλο και ένα για να οργανώσουμε τη βάση με βάση το επίθετο του συγγραφέα. Στην περίπτωση του δεύτερου, επιστρέφεται ως αποτέλεσμα μια λίστα από όλα τα βιβλία στα οποία αναφέρεται κάποιος ως συγγραφέας, είτε μόνος του είτε ως συνεργάτης με άλλους συγγραφείς.

\section{Μέρος Δ}
Στο module AVL.py είναι υλοποιημένο ένα κομβοπροσανατολισμένο AVL δέντρο οργανωμένο με βάση τα ids των βιβλίων.Σε κάθε πράξη ελέγχεται η ζύγιση του υποδέντρου και αν έχει χαλάσει γίνονται κατάλληλες επαναζυγιστικές κινήσεις. Σε αυτο πρόσθέτουμε τη βάση των βιβλίων.

\section{Μέρος Ε}
Για το ερώτημα αυτό χρησιμοποιείται το αρχείο test.py σε συνδιασμό με το create.py. Το αρχείο create.py αρχικά δημιουργεί μια λίστα από βιβλία τα οποία και θα αναζητήσουμε και το αποθηκεύει μέσω του pickle module για εύκολη αργότερη χρήση.
Έτσι στο test.py χρησιμοποιούμε αυτή τη λίστα από βιβλία δείγματα ώστε να αναζητήσουμε πάνω στη βάση των βιβλίων. Το μέγεθος της βάσης το μεταβάλουμε ώστε να δούμε την απόδοση των αλγορίθμων ανάλογα με το μέγεθος και για κάθε μέγεθος βάσης παίρνουμε το μέσο χρόνο αναζήτησης.
Για να είναι πιο αντικειμενικές οι μετρήσεις, όσο μεγαλύτερη είναι η βάση μας κάθε φορά τόσες περισσότερες αναζητήσεις εκτελούμε πάνω της.

Τελικά χρησιμοποιούμε το module matplotlib ώστε να οπτικοποιήσουμε αυτά τα αποτελέσματα και να είναι πιο εύκολο να καταλάβουμε την διαφορά της απόδοσης τους ανάλογα με το μέγεθος της βάσης που χρησιμοποιούμε.
Παρατηρούμε ότι η απόδοση των γραμμικών αναζητήσεων πέφτει πολύ γρήγορα όσο μεγαλώνει το μέγεθος της βάσης, σε αντίθεση με τις αναζητήσεις που έχουν ως βάση AVL ή TRIES καθώς και την binary search. Αυτές διατηρούν πολύ καλή απόδοση ακόμη και παρά τη μεγάλη αύξηση του μεγέθους της βάσης των βιβλιών. Αυτό φαίνεται και στα screenshot που παρατήθονται και κάνουν σύγκριση των χρόνων αναζήτησης ανάμεσα στις διαφορετικές υλοποιήσεις.

\subsection{Αναζήτηση με βάση το ID}
\includegraphics[scale=0.6]{id.png}
Εδώ παρατηρούμε ότι όπως αναμέναμε η γραμμική αναζήτηση αυξάνει σχεδόν γραμμικά με την αύξηση του μεγέθους της βάσης των βιβλίων. Προφανώς και πάλι υπάρχουν κάποιες αυξομειώσεις καθώς κάποιες αναζητήσεις μπορεί να σταθούν "τυχαιρές" και έτσι δεν έχουμε τη γραφική παράσταση μιας τέλειας ευθείας.

Αντίθετα, τόσο η δυαδική αναζήτηση όσο και η αναζήτηση με το AVL δέντρο κινούνται σε λογαριθμικούς χρόνους και η αύξηση του χρόνου αναζήτησης με αυτές τις δύο είναι μικρή καθώς αυξάνει η βάση των βιβλιών μας.
Αυτό συμφωνέι με τη θεωρία που μας λέει ότι η δυαδική αναζήτηση έχει λογαριθμικό μέσο χρόνο αναζήτησης. Το ίδιο complexity έχει και το AVL, το οποίο και αυτό επαληθεύεται μέσω της γραφικής αυτής παράστασης.
 
\subsection{Αναζήτηση με βάση το title}
\includegraphics[scale=0.6]{title.png}
Εδώ συγκρίνουμε τη γραμμική αναζήτηση μέσω τίτλου με την αναζήτηση μέσω ψηφιακών δέντρων (TRIES). Βλέπουμε ότι ενώ η γραμμική αναζήτηση αυξάνει τους χρόνους της όσο αυξάνεται η βάση των βιβλίων, η αναζήτηση μέσω των TRIE μένει σχεδόν σταθερή.
Αυτό δεν πρέπει να μας εκπλήσει, καθώς η αναζήτηση μέσω TRIES εξαρτάται κυρίως από το μέγεθος του εκάστοτε τίτλου και όχι άμεσα από τη βάση των βιβλίων.
Αυτό έχει ως αποτέλεσμα κάποια entries, ιδιαίτερα αυτά που ξεκινούν από κάποιο πιο "σπάνιο" prefix να έχουν πολύ γρήγορους χρόνους αναζήτησης. Μπορούμε να πούμε ότι γενικά τα TRIE έχουν πάρα πολύ καλή απόδοση στην αναζήτηση των βιβλίων, που δεν εξαρτάται τόσο από το μέγεθος της βάσης πάνω στην οποία αναζητούμε αλλά μόνο από την ίδια τη πληροφορία των βιβλίων.
Αυτό θα το επαληθεύσουμε και με το παρακάτω τελευταίο πείραμα.
\subsection{Αναζήτηση με βάση τους author}
\includegraphics[scale=0.6]{authors.png}
\end{document}



